\documentclass{article}
\usepackage[utf8]{inputenc}
\usepackage{amsmath}
\usepackage{graphicx}
\usepackage{physics}
\usepackage{listings}
\begin{document}
\section{Problem 2 within project 1}
We are looking to write a program that defines a vector of x-values, and a function that evaluates the excact solution over these values, the results of which
will be stored in 2 columns, with a fixed amount of decimals, in scientific notation. This data will also be plotted  using a separate script.
\subsection*{The function}
we've implemented a function that returns a double $u(x)$ for an argument $x$, who's type is double, such that:
\begin{align}    
    \label{eq:excact_solution}
    u(x) =  1 - (1-e^{-10})x - e^{-10x}
\end{align}
The function "analytic\_sol" is declared as the type double, and so is it's (only) parameter "x" too. the function body evaluates $u(x)$ as shown in \ref{eq:excact_solution}, and returns the
result. Vectors x and u were initialized to contain 101 elements doubles, and x was fully defined by assigning each element within a for-loop, iterating through the elements the elements and
assigning each one a value. (each element is assigned the value of the previous element plus 1/100, starting at 0, such that $x_{i+1} = x_i$). u was defined using a for loop, 
calling "analytic\_sol" with $x_i$ as the input, such that $u_{i} = \text{analytic\_sol}(x_i)$.

x is a vector containing 101 linearly spaced doubles ranging from 0 to 1, and u is similarily a vector of doubles, such that $u_i=u(x_i)$. The elements of these vectors are stored in a .dat file,
and loaded using python, for which the data is used to initialize numpy arrays, for the purpose of displaying u(x). The results of which can be seen in the figure:
\begin{figure}[h]
    \includegraphics{problem_2_fig.pdf}
    \caption{Shows 101 linearly spaced points starting at 0 and ending at 1, evaluated on the excact solution $u(x)=1 - (1 - e^{-10})x - e^{-10x}$.}
\end{figure}
\end{document}