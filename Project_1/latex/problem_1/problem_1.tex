\documentclass{article}
\usepackage[utf8]{inputenc}
\usepackage{amsmath}
\usepackage{physics}
\title{Problem 1 within project 1}
\begin{document}
\maketitle
We are interested in the solution for
\begin{align*}
    -\dv[2]{u}{x}  = 100 e^{-10x}
\end{align*}
It is claimed that $u(x)=1-(1-e^{-10})x-e^{-10x}$ is a solution. Inserting this into the lefthand side of the original expression will show that this is indeed a solution if both sides
of the equation still match:
\begin{align*}
     & -\dv[2]{}{x}\left(1-(1-e^{-10})x-e^{-10x}\right) \\
     & =-(-100e^{-10x})=100e^{-10x}
\end{align*}
As such, we've shown that $u(x)=1-(1-e^{-10})x-e^{-10x}$ is an exact solution, meaning that this solution holds $\forall x$.
\end{document}